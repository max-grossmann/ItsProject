\subsection{Allgemeines}\label{subsec:allgemeines}

Die für diese Schwachstelle benutzte Version von ImageMagick ist Version 6.9.3-9.\\
Das verwendete Betriebssystem ist Ubuntu 16.04.\\

ImageMagick kommt mit einigen Standardbefehlen.
Die meist verwendeten davon sind "`convert"' und "`identify"'.\\\\
"`convert"' konvertiert ein Bild in ein anderes Dateiformat, "`identify"' analysiert das Bild und gibt Informationen zurück.\\

Wird zum Beispiel der Befehl "`"convert"' im Terminal übergeben, mappt das Linux-System diesen anhand der verknüpften Binaries.\\

Die verknüpften Binary-Dateien können im Linux-Terminal mit dem Befehl "`whereis"' abgefragt werden:\\

\begin{lstlisting}[language=Text, caption=whereis Binary Abfrage,label={lst:lstlisting}]
root@vm-its:/home/max# whereis convert
convert: /usr/local/bin/convert
\end{lstlisting}
\vspace{5mm}

Die verwendete Binary zum Starten von ImageMagick mit "`convert"' ist also in \\/usr/local/bin/convert.\\
Der Ablauf für andere Befehle wie identify ist analog.\\