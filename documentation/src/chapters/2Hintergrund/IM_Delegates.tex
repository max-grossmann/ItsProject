\subsection{Delegates}\label{subsec:delegates}

ImageMagick unterstützt das Bearbeiten von Bildern mit Hilfe diverser Quellen und Formaten.\\
Diese sind als “Delegates” implementiert und dadurch austauschbar.\\

Nativ wird ImageMagick mit drei verschiedenen Arten von Delegates ausgeliefert, die sich durch die Angabe des Eingabe- beziehungsweise Ausgabeformats unterscheiden:
\begin{itemize}
    \item Translatoren spezifizieren das Eingabe- und Ausgabeformat.
    \item Dekoder spezifizieren nur das Eingabeformat.\\
    Das Ausgabeformat wird automatisch erkannt.
    \item Enkoder spezifizieren nur das Ausgabeformat.\\
    Das Eingabeformat wird automatisch erkannt.
\end{itemize}

Innerhalb der Delegates werden dann “system()”-Aufrufe verwendet, um die Arbeit an die Shell weiterzuleiten.\\

In der Datei “delegates.xml” findet das Mapping zwischen Delegate und spezifischem Shell-Befehl statt.\\

[Bild bencode.net delegates]