\newpage
\subsection{Delegates}\label{subsec:delegates}

ImageMagick unterstützt das Bearbeiten von Bildern mithilfe diverser Quellen und Formaten.\\
Diese sind als "`Delegates"' implementiert und dadurch austauschbar.\\

Nativ wird ImageMagick mit drei verschiedenen Arten von Delegates ausgeliefert,
die sich durch die Angabe des Eingabe- beziehungsweise Ausgabeformats unterscheiden:
\begin{itemize}[\itemsep=1em]
    \item Translatoren spezifizieren das Eingabe- und Ausgabeformat.
    \item Dekoder spezifizieren nur das Eingabeformat.\\
    Das Ausgabeformat wird automatisch erkannt.
    \item Enkoder spezifizieren nur das Ausgabeformat.\\
    Das Eingabeformat wird automatisch erkannt.
\end{itemize}

Innerhalb der Delegates werden dann "`system()"'-Aufrufe verwendet, um die Arbeit an die Shell weiterzuleiten.\\

In der Datei "`delegates.xml"' findet das Mapping zwischen Delegate und spezifischem Shell-Befehl statt.

\begin{lstlisting}[language=XML, caption=/config/delegates.xml.in Auszug,label={lst:lstlisting}]
<delegatemap>
  <delegate decode="png" encode="bpg" command="&quot;@BPGEncodeDelegate@&quot; -b 12 -q %[fx:quality/2] -o &quot;%o&quot; &quot;%i&quot;"/>
  <delegate decode="browse" stealth="True" spawn="True" command="&quot;@BrowseDelegate@&quot; http://www.imagemagick.org/; rm &quot;%i&quot;"/>
  <delegate decode="cdr" command="&quot;@UniconvertorDelegate@&quot; &quot;%i&quot; &quot;%o.svg&quot;; mv &quot;%o.svg&quot; &quot;%o&quot;"/>
  <delegate decode="cgm" command="&quot;@UniconvertorDelegate@&quot; &quot;%i&quot; &quot;%o.svg&quot;; mv &quot;%o.svg&quot; &quot;%o&quot;"/>
  <delegate decode="dot" command='&quot;@GVCDecodeDelegate@&quot; -Tsvg &quot;%i&quot; -o &quot;%o&quot;' />
  <delegate decode="edit" stealth="True" command="&quot;@EditorDelegate@&quot; -title &quot;Edit Image Comment&quot; -e vi &quot;%o&quot;"/>
  <delegate decode="html" command="&quot;@HTMLDecodeDelegate@&quot; -U -o &quot;%o&quot; &quot;%i&quot;"/>
  <delegate decode="https" command="&quot;@WWWDecodeDelegate@&quot; -s -k -L -o &quot;%o&quot; &quot;https:%M&quot;"/>
</delegatemap>
\end{lstlisting}
\vspace{5mm}

Der letzte Delegate "`https"' wird sich im Folgenden als mitverantwortlich für die Schwachstelle zeigen.