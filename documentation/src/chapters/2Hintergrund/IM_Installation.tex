\subsection{Installation}\label{subsec:installation}

Für die Installation wird ein Linux Server mit Ubuntu 16.04 aufgesetzt.
Dieser Server wird in diesem und in den folgenden Kapiteln zur Einrichtung benutzt.\\

Imagemagick wird in dem folgenden Kapitel über dessen Source-Code compiliert und installiert.
Dies hat den Vorteil, dass eine explizite Version gewählt werden kann.\\

Außerdem ist die Version, die von Ubuntu 16.04 über die Paketquellen ausgeliefert wird zu neu,
um die Schwachstelle auszunutzen.\\

Zuerst müssen folgende Dependencies installiert werden, um C-Anwendungen kompilieren zu können:

\begin{lstlisting}[language=Bash, caption=Imagemagick Installation: Dependencies,label={lst:installdep}]
apt install make
apt install gcc
\end{lstlisting}
\vspace{5mm}

Weiter wird nun eine von der Schwachstelle betroffene Imagemagick Version heruntergeladen und entpackt.
\begin{lstlisting}[language=Bash, caption=Imagemgaick Installation: Source Code herunterladen und entpacken,label={lst:installsource}]
wget https://www.imagemagick.org/download/releases/ImageMagick-6.9.2-10.tar.xz
tar xvf ImageMagick-6.9.2-10.tar.xz
cd ImageMagick-6.9.2-10.tar.xz
\end{lstlisting}
\vspace{5mm}

Als nächstes muss das configure-Script ausgeführt werden.
Dieses ist im Imagemagick Sourcecode enthalten.
Das Configure Script ist unter anderem dafür verantwortlich zu überprüfen,
ob ein C compiler und alle sonstigen benötigten Abhängigkeiten auf dem System installiert sind
~\cite{georgebrocklehurstMagicConfigureMake}.\\\\

\begin{lstlisting}[language=Bash, caption=Imagemagick Installation: Configure,label={lst:installconfigure}]
./configure

\end{lstlisting}
\vspace{5mm}

\newpage
Für das direkte Ausnutzen der Schwachstelle wird kein JPEG und PNG Support benötigt.
Allerdings ist es für demonstrationszwecke sinnvoll, auch diese Dateiformate aktiviert zu haben.
Auch in der PHP Anwendung soll man später die Möglichkeit haben selbst solche Bild-Dateien hochladen zu können,
auf denen dann Imagemagick-Operationen angewendet werden.\\\\

Beim betrachten der Configure-Ausgabe, wird ersichtlich,
dass zwar versucht wird die Delegates für PNG und JPEG zu laden (siehe Option),
dies aber nicht möglich ist. (Siehe 'no' in 'Value' Spalte)

\begin{lstlisting}[language=Text, caption=Imagemagick Installation: Auszug aus Configure-Output,label={lst:installconfigureoutput}]
Option                        Value
-------------------------------------------------------
(...)

Delegate Library Configuration:
(...)
JPEG v1           --with-jpeg=yes             no
(...)
PNG               --with-png=yes              no
(...)
\end{lstlisting}
\vspace{5mm}

Das hat den Hintergrund, dass noch Abhängigkeiten für die expliziten Dateitypen fehlen.\\\\

Will man z.B.\ auch das JPEG installieren können, muss folgendes Paket noch installiert werden~\cite{ImageMagickPNGDelegate}:

\begin{lstlisting}[language=Bash, caption=Imagemagick Installation: Delegate Dependencies,label={lst:installdelegatedep}]
apt install libjpeg-dev
\end{lstlisting}
\vspace{5mm}

Führt man nun den Configure-Command erneut aus, steht der Wert in der Value-Spalte auf yes.\\

In der Regel installiert man die Abhängigkeiten für alle Formate, die Imagemagick unterstützt.
Dies wird möglich, in dem man die Debian Quellpakete
~\cite{DateiEtcApt} in der sources-list aktiviert.~\cite{HowInstallImageMagick}\\


\begin{lstlisting}[language=Bash, caption=Imagemagick Installation: Sources List,label={lst:installsourcelist}]
vim /etc/apt/sources.list
deb-src http://de.archive.ubuntu.com/ubuntu/ xenial main restricted
\end{lstlisting}
\vspace{5mm}

\newpage
Bevor die Build-Dependencies installiert werden können, müssen noch die Paketquellen geupdatet werden:

\begin{lstlisting}[language=Bash, caption=Imagemagick Installation: Imagemagick Build Dependencies,label={lst:installimbuilddeps}]
apt update
apt build-dep imagemagick
\end{lstlisting}
\vspace{5mm}

Führt man nun configure erneut aus, sieht man, dass nun sämtliche Delegates aktiviert werden.
Darunter auch PNG und JPEG.
\begin{lstlisting}[language=Text, caption=Imagemagick Installation: Auszug Configure Output V2,label={lst:installoutput2}]
Delegate Library Configuration:
  BZLIB             --with-bzlib=yes            yes
  Autotrace         --with-autotrace=no         no
  DJVU              --with-djvu=yes             yes
  DPS               --with-dps=yes              no
  FFTW              --with-fftw=yes             yes
  FlashPIX          --with-fpx=yes              no
  FontConfig        --with-fontconfig=yes       yes
  FreeType          --with-freetype=yes         yes
  Ghostscript lib   --with-gslib=no             no
  Graphviz          --with-gvc=yes              no
  JBIG              --with-jbig=yes             yes
  JPEG v1           --with-jpeg=yes             yes
  LCMS              --with-lcms=yes             yes
  LQR               --with-lqr=yes              yes
  LTDL              --with-ltdl=yes             no
  LZMA              --with-lzma=yes             yes
  Magick++          --with-magick-plus-plus=yes yes
  OpenEXR           --with-openexr=yes          yes
  OpenJP2           --with-openjp2=yes          no
  PANGO             --with-pango=yes            yes
  PERL              --with-perl=no              no
  PNG               --with-png=yes              yes
  RSVG              --with-rsvg=no              no
  TIFF              --with-tiff=yes             yes
  WEBP              --with-webp=yes             no
  WMF               --with-wmf=yes              yes
  X11               --with-x=                   yes
  XML               --with-xml=yes              yes
  ZLIB              --with-zlib=yes             yes
\end{lstlisting}
\vspace{5mm}

Um Imagemagick nun zu bauen, muss der make-Befehl aufgerufen werden~\cite{HowInstallImageMagick}.

\begin{lstlisting}[language=Bash, caption=Imagemagick Installation: make Befehl,label={lst:installmake}]
make
\end{lstlisting}
\vspace{5mm}

Abschließend müssen die Imagemagick-Scripte noch installiert werden.
Dabei werden diese an Orte kopiert, von denen sie global aufgerufen werden können~\cite{HowInstallImageMagick}.

\begin{lstlisting}[language=Bash, caption=Imagmagick Installation: make install Befehl,label={lst:installmakeinstall}]
make install
\end{lstlisting}
\vspace{5mm}