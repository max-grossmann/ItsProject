\section{Shell Grundlagen}\label{sec:shell-grundlagen}

Eine Shell ist ein Programm, das die Steuerung des Computers über die Kommandozeile ermöglicht.\\

Dabei werden Befehle mit der Tastatur eingegeben, an statt über ein graphisches Interface mit Maus und Keyboard.\\

Als ursprüngliche Art und Weise, mit einem Computer zu arbeiten, bietet die Shell einige Vorteile:
\begin{itemize}
    \item Viele Informationen, die sonst nur durch zahlreiche Klicks mit der Maus erreichbar sind, sind über die Shell mit einem Befehl abrufbar.
    \item Das Automatisieren von Aufgaben ist deutlich einfacher.
    \item Die Arbeit ist einfacher reproduzierbar.
    \item Externe Server können meistens nur über eine Shell benutzt werden.
\end{itemize}

Die Shell ist auf Linux und MacOS als Terminal vorinstalliert.
Auf Windows wird ein separates Programm benötigt.\\

Nach dem Starten des Terminals werden auf Linux Systeminformationen angezeigt und der aktuelle Pfad, in dem die Shell arbeitet.\\

[Bild]\\

Nun stehen bestimmte Standardbefehle (mit Ergänzungen) zur Verfügung:
\begin{itemize}
    \item ls: Listet alle Dateien und Ordner im aktuellen Verzeichnis auf.
    \item cd: wechselt in das angegebene Verzeichnis
    \item ALLES, WAS SPÄTER NOCH KOMMT
\end{itemize}

Um einzelne Befehle verketten zu können, wird in der Shell die Pipe “|” verwendet.\\
Dabei wird die Ausgabe des ersten Befehls als Eingabe für den nächsten Befehl verwendet.
