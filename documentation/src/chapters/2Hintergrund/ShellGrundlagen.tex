\section{Shell Grundlagen}\label{sec:shell-grundlagen}

Eine Shell ist ein Programm, das die Steuerung des Computers über die Kommandozeile ermöglicht.\\

Dabei werden Befehle mit der Tastatur eingegeben, an statt über ein graphisches Interface mit Maus und Keyboard.\\

Als ursprüngliche Art und Weise, mit einem Computer zu arbeiten, bietet die Shell einige Vorteile:
\begin{itemize}[\itemsep=1em]
    \item Viele Informationen, die sonst nur durch zahlreiche Klicks mit der Maus erreichbar sind, sind über die Shell mit einem Befehl abrufbar.
    \item Das Automatisieren von Aufgaben ist deutlich einfacher.
    \item Die Arbeit ist einfacher reproduzierbar.
    \item Externe Server können meistens nur über eine Shell benutzt werden.
\end{itemize}

Die Shell ist auf Linux und MacOS als Bash im Terminal vorinstalliert.
Auf Windows läuft die Shell als PowerShell.
Zur Benutzung von Bash-Kommandos wird ein separates Programm benötigt.\\
In dieser Arbeit wird auf Linux gearbeitet, also mit Bash.\\

\newpage

Nach dem Starten des Terminals werden auf Linux Systeminformationen angezeigt und der aktuelle Pfad, in dem die Shell arbeitet.

\begin{lstlisting}[language=Bash, caption=Beispiel Bash Start,label={lst:lstlisting}]
root@vm-its:/home/max# ...
\end{lstlisting}
\vspace{5mm}

Nun stehen bestimmte Standardbefehle (mit Ergänzungen) zur Verfügung:
\begin{itemize}[\itemsep=1em]
    \item ls: Listet alle Dateien und Ordner im aktuellen Verzeichnis auf.
    \item cd: wechselt in das angegebene Verzeichnis
    \item wget/curl: Wird benutzt, um Dateien herunterzuladen~\cite{WgetRef}.
    \item echo: Ausgabe von Zeichenketten und Variablen auf dem Standardausgabegerät~\cite{EchoCmd}
    \item cat: War ursprünglich zum Zusammenfügen von Dateien gedacht, wird jetzt aber häufig nur benutzt, um den Inhalt einer Datei anzuzeigen~\cite{CatRef}.
\end{itemize}

Um einzelne Befehle verketten zu können, wird in der Shell die Pipe "`|"' verwendet.\\
Dabei wird die Ausgabe des ersten Befehls als Eingabe für den nächsten Befehl verwendet.
