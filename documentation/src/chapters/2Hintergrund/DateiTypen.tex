\section{Datei-Typen}\label{sec:datei-typen}

Umgangssprachlich werden als "`Magic Bytes"' die ersten Bytes einer Datei bezeichnet.\\
Diese werden verwendet, um den Dateitypen zu identifizieren.\\

Diese Dateisignatur ist beim Öffnen einer Datei nicht sichtbar, kann aber zum Beispiel mit dem Linux-CommandLine-Tool "`xxd"' angezeigt werden.\\

\begin{lstlisting}[language=Bash, caption=Magic Bytes XXD,label={lst:lstlisting}]
max@vm-its:~$ xxd test.jpeg
00000000: ffd8 ffdb 0084 0003 0202 0302 0203 0303  ................
...
\end{lstlisting}
\vspace{5mm}

In einer .jpg-Datei sind die Magic Bytes immer "`FF D8 FF DB"', in einer .zip-Datei "`50 4B 03 04"'.\\
So kann eine Datei sehr leicht und schnell identifiziert werden, ohne den kompletten Inhalt laden zu müssen, und das passende Programm zum Verarbeiten der Datei geöffnet werden.