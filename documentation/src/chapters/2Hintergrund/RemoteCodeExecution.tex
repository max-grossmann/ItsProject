\section{Remote Code Execution}\label{sec:remote-code-execution}

Über eine "`Remote-Code-Execution"', kurz "`RCE"', kann ein Angreifer auf einen fremden Computer zugreifen und diesen steuern.\\

Dabei muss er dazu nicht autorisiert sein oder auch nur in der Nähe des angregriffenen Computers sein.\\

Über eine "`RCE"' kann der Angreifer schadhaften Code ausführen und so Daten klauen, den Rechner des Opfers beeinflussen oder im schlimmsten Fall sogar unnutzbar machen.\\

"`Remote-Code-Executions"' gehören zu den verheerendsten Schwachstellen in IT-Systemen, da der Angreifer praktisch keine Einschränkungen hat.
