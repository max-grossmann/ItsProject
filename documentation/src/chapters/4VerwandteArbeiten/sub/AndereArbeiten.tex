\section{Andere Arbeiten zu der CVE-2016-3714}\label{sec:arbeiten-zu-der-cve-2016-3714}

\subsection{Oracle Linux Bulletin - April 2016~\cite{OracleLinuxBulletin}}\label{subsec:oracle-linux-bulletin---april-2016}

Der Oracle Linux Bulletin listet alle Sicherheitslücken mit dazugehöriger CVE auf,
welche aktuell in Oracle Linux existieren.
Die Liste erscheint gleichzeitig mit dem Security Patch.
In der Ausgabe von April 2016 ist auch hier behandelte CVE 2016-3714 enthalten.


\subsection{OpenSuse Mailing-Liste - Mai 2016~\cite{SecurityannounceSUSESU201612751}}\label{subsec:opensuse-mailing-liste---may-2016}

Per Mailing-List werden Security-Updates für Suse Produkte angekündigt.
Darunter befindet sich auch die hier beschriebene Sicherheitslücke.
Suse deaktiviert zusätzlich einige Coder, welche für remote code execution attacks anfällig sind.
Diese können per Umgebungsvariable wieder aktiviert werden.
In der Mailingliste werden auch für verschiedene Suse Produkte beschrieben,
welche Schritte konkret durchgeführt werden müssen, um die Sicherheitslücke zu schließen


\subsection{Exploit Database - Erklärung~\cite{ExploitDBErklaerung}}\label{subsec:exploit-database:-erklaerung}

Die Website exploit-db.com sammelt Informationen zu öffentlich zugänglichen exploits und archiviert diese~\cite{ExploitDBFaq}.
Exploits können außerdem von jeder Person eingereicht werden~\cite{ExploitDbSubmit}.\\

Auch zu der hier vorliegenden CVE ist ein Eintrag in der Exploit-DB vorhanden.
Es wird beschrieben, wie die Code Execution anhand von dem Imagemagick "convert" durchgeführt werden kann.
Es wird auch darauf eingegangen, dass angreifender Code in Dateien,
wie MVG und SVG platziert werden kann und somit die Sicherheitslücke nochmal gefährlicher wird.


\subsection{Imagemagick Forum~\cite{ImageMagickSecurityIssue}}\label{subsec:imagemagick-forum}

Die Forum Ankündigung beschriebt,
dass gewisse Coder (darunter auch HTTPS) angreifbar für Remote Code Execution Attacks sind.
Es werden außerdem Konfigurationsmöglichkeiten aufgezeigt,
die die Sicherheit von Imagemagick erhöhen und damit solche Angriffe verhindern.