\newpage
\section{Verwandte CVEs}\label{sec:verwandte-cves}

Die hier beschriebene CVE 2016-3714 ist Teil von ImageTragick~\cite{ImageTragick}, einer Sammlung von Vulnerabilities
innerhalb der Software ImageMagick.
Alle Vulnerabilites wurden von Nikolay Ermishkin aus dem Mail.Ru Security Team aufgedeckt~\cite{ImageTragick}.
Die Website ImageTragick ist mit dem Hintergedanken entstanden,
mehr Menschen auf die Sicherheitslücken von ImageMagick und deren Vermeidung aufmerksam zu machen~\cite{ImageTragick}.\\

Zu ImageTragick gehören zusätzlich zu CVE 2016-3714 noch folgende Sicherheitslücken:

\begin{itemize}
    \item CVE-2016-3718~\cite{CVE20163718HTTPFTP}: Mithilfe MVG Dateien können HTTP und FTP Requests abgesetzt werden
    \item CVE-2016-3715~\cite{CVE20163715EPHEMERALCoder}: Durch das 'ephemeral'-Protokoll ist es möglich Dateien auf dem Zielsystem zu löschen
    \item CVE-2016-3716~\cite{CVE20163716MSLCoder}: Durch das 'msl'-Protokoll ist es möglich Dateien auf dem Zielsystem zu verschieben
    \item CVE-2016-3717~\cite{CVE20163717LABELCoder}: Durch das 'label'-Protokoll kann der Inhat von Dateien ausgelesen und auf das Bild platziert werden.
\end{itemize}