\section{Schwachstelle}\label{sec:schwachstelle}

Die in dieser Studienarbeit behandelte Schwachstelle ist eine dieser "`Remote-Code-Execution"' -Schwachstellen.\\

Die CVE-2016-3714 beschreibt die Möglichkeit, mittels eines mit Shell-Befehlen präparierten Bildes, fremden Code auf dem angegriffenen Rechner auszuführen.\\

Diese Sicherheitslücke ist so bekannt, dass sie inzwischen als Teil einer Sammlung von Sicherheitslücken unter dem Namen "`ImageTragick"' bekannt ist.\\

ImageTragick hat den höchsten CVSS-Score von 10, was vor allem an der einfachen Durchführung des Angriffs und der möglichen weitreichenden Beeinflussung des angegriffenen Systems liegt~\cite{ImagemagickProductsVulnerabilities}.\\
Das präparierte Bild ist mit drei Codezeilen geschrieben und muss nur zum Beispiel an den anzugreifenden Webserver geschickt werden.\\
Der Angreifer kann über die im Bild integrierten Shell-Befehle sämtliche Befehle auf User-Recht-Level ausführen.\\

Betroffen von dieser Schwachstelle sind die ImageMagick-Versionen vor 6.9.3-10 sowie Versionen 7.x vor 7.0.1-1.\\
Folgende OS-Versionen sind von der Schwachstelle betroffen~\cite{CVE20163714EPHEMERALHTTPS}:
\begin{itemize}
    \item Ubuntu 12.04
    \item Ubuntu 14.04
    \item Ubuntu 15.10
    \item Ubuntu 16.04
    \item Debian 8.0
    \item Debian 9.0
    \item Leap 42.1
    \item Opensuse 13.2
    \item Suse Linux Enterprise Server 12
\end{itemize}
Gerade auf älteren Systemen, die nicht regelmäßig gewartet oder geupdatet werden, besteht eine realistische Chance, dass eine der betroffenen ImageMagick-Versionen noch läuft.\\