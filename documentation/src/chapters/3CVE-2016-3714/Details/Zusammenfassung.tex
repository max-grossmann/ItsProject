\subsection{Zusammenfassung}\label{subsec:zusammenfassung}

ImageMagick bietet die Möglichkeit eine Bild-Datei per URL herunterzuladen und zum Beispiel Konvertierungsoperationen auf
dieser anzuwenden.\\

Folgendes Beispiel lädt ein Bild von "`\%BILD\_URL\%"' herunter, konvertiert es zu einer JPG Datei und speichert es unter dem Namen "`image.jpg"' auf dem Dateisystem ab.
\begin{lstlisting}[language=Bash, caption=Beispielbefehl Codeablauf,label={lst:codeablaufbeispiel}]
convert '%BILD_URL%' image.jpg
\end{lstlisting}
\vspace{5mm}

Besonders gefährlich wird es, wenn URLs in MVG bzw.
SVG Dateien eingebettet sind.
Speziell präparierte MVG Dateien können dann unter anderem von remote an einen Webserver übergeben werden.
Führt dieser dann ImageMagick-Operationen aus, kann Remote Code ausgeführt werden.
Ausführliche Beispiele sind im Kapitel "`Ausnutzung der Schwachstelle"' zu finden.\\

Um die Bilddaten der URL zu bekommen, wird der externe https-Delegate benutzt.\\

Der https-Delegate benutzt für das Mapping den folgenden Befehl, wie in der Datei config/delegates.xml.in ersichtlich ist~\cite{DelegatesXml}.\\

\begin{lstlisting}[firstnumber=90, language=XML, caption=config/delegates.xml.in https-Delegate,label={lst:lstlisting}]
  <delegate decode="https" command="&quot;@WWWDecodeDelegate@&quot; -s -k -L -o &quot;%o&quot; &quot;https:%M&quot;"/>
\end{lstlisting}
\vspace{5mm}

Löst man "`\&quot"' nun zu Anführungszeichen auf, ergibt sich der Befehl:\\
\begin{lstlisting}[firstnumber=1, language=Bash, caption=Aufgelöster https-Delegate-Befehl,label={lst:lstlisting}]
"@WWWDecodeDelegate@" -s -k -L -o "%o" "https:%M"
\end{lstlisting}
\vspace{5mm}

Der "`@WWWDecodeDelegate@"' ist dabei der systemspezifische Befehl für das Herunterladen einer Datei aus dem Internet - für Linux also wget, bzw. curl.\\
Im Platzhalter "`\%M"' liegt später die URL, verknüpft mit dem Bash-Angriffsbefehl.\\

\newpage
Genau hier liegt die Schwachstelle.
Der URL-Parameter wird nicht ausreichend gesanatized, bevor er mit dem Platzhalter "`\%M"' ersetzt und der neue Befehl an den system()-Call weitergegeben wird.\\

Es wird nun folgende URL betrachtet, zusammen mit dem Angriffscode:

\begin{lstlisting}[firstnumber=91, language=Bash, caption=Beispielhafte URL mit Angriffscode,label={lst:lstlisting}]
https://example.com"|ls "-la
\end{lstlisting}
\vspace{5mm}

Setzt man die URL nun in den "`\%M"' Parameter des HTTPS-Delegates ein, ergibt sich unter Linux folgender vereinfachter Befehl:

\begin{lstlisting}[language=Bash, caption=HTTPS Delegate mit Angriffscode,label={lst:angriffscodedelegate}]
"curl" "https:https://example.com"|ls "-la"
\end{lstlisting}
\vspace{5mm}

ImageMagick nimmt nun diesen Befehl, erkennt die URL, lädt das Bild herunter und verarbeitet es wie gewünscht.\\

Dabei wird der "`angehängte"' Angriffscode aber einfach mitgegeben bis zur Shell.
Man erkennt, dass der "`ls -la"'-Teil nicht mehr Bestandteil der URL, sondern als eigenständiger Befehl - verknüpft mit der Linux-Pipe - hinter dem URL-Argument steht.\\

Die Shell interpretiert die Pipe "`|"' standardmäßig als Verknüpfung.
Die Ausgabe des ersten Befehls wird als Input des zweiten Befehls verwendet.\\
Da der zweite Befehl keinen Input benötigt, wird dieser einfach im Hintergrund ausgeführt und der Angriff ist erfolgreich.\\