\subsection{Fix ImageMagick 6.9.3-10}\label{subsec:fix-imagemagick-6.9.3-10}

Die Schwachstelle ist in Version 6.9.3-10 beseitigt.\\
Im Folgenden werden die einzelnen Codestellen erläutert, die dafür verantwortlich sind.\\

In der InvokeDelegate()-Methode in der Datei magick/delegate.c, die wie oben erwähnt, eingebettete Zeichen mit den dazugehörigen Bildattributen ersetzt, wird auch der folgende Befehl aufgerufen:\\

\begin{lstlisting}[firstnumber=1295, language=C, caption=magick/delegate.c Aufruf InterpretImageProperties(),label={lst:lstlisting}]
command=InterpretImageProperties(image_info,image,commands[i]);
\end{lstlisting}
\vspace{5mm}

Die InterpretImageProperties()-Methode befindet sich in magick/property.c und übernimmt ebendiesen Austausch von Zeichen mit Bildattribut:\\

\begin{lstlisting}[firstnumber=3347, language=C, caption=magick/property.c InterpretImageProperties(),label={lst:lstlisting}]
MagickExport char *InterpretImageProperties(const ImageInfo *image_info,
  Image *image,const char *embed_text)
{
  ...
}
\end{lstlisting}
\vspace{5mm}

Dabei liest sie den "`MagickPropertyLetter"' aus:\\

\begin{lstlisting}[firstnumber=3466, language=C, caption=magick/property.c Aufruf GetMagickPropertyLetter(),label={lst:lstlisting}]
  value=GetMagickPropertyLetter(image_info,image,*p);
\end{lstlisting}
\vspace{5mm}

Die GetMagickPropertyLetter()-Methode such das spezifische Attribut für einen Character:\\

\begin{lstlisting}[firstnumber=2343, language=C, caption=magick/property.c GetMagickPropertyLetter(),label={lst:lstlisting}]
static const char *GetMagickPropertyLetter(const ImageInfo *image_info,
  Image *image,const char letter)
{
  ...
}
\end{lstlisting}
\vspace{5mm}

Da, wie erwähnt, der Übergabeparamter beim https-Delegate "`M"' ist, wird dieser angewendet:\\

\begin{lstlisting}[firstnumber=2627, language=C, caption=magick/property.c Ungefilterte Weitergabe M-Parameter,label={lst:lstlisting}]
  case 'M':
  {
    /*
      Magick filename - filename given incl. coder & read mods.
    */
    string=image->magick_filename;
    break;
  }
\end{lstlisting}
\vspace{5mm}

Hier ist ersichtlich, dass, was auch immer in "`M"' stehen mag, ungefiltert und ohne Überprüfung weitergericht wird.\\

Die Verteidigung der Schwachstelle ist ab Version 6.9.3-10 so gelöst, dass für das https-Delegate anstatt Parameter "`M"' der neu eingeführte Parameter "`F"' verwendet wird:\\

\begin{lstlisting}[firstnumber=91, language=XML, caption=config/delegates.xml.in https-Delegate 6.9.3-10,label={lst:lstlisting}]
  <delegate decode="https" command="&quot;@WWWDecodeDelegate@&quot; -s -k -L -o &quot;%o&quot; &quot;https:%F&quot;"/>
\end{lstlisting}
\vspace{5mm}

Bereinigt um die Umschreibung der Anführungszeichen ergibt sich:\\

\begin{lstlisting}[firstnumber=1, language=Bash, caption=Aufgelöster https-Delegate-Befehl 6.9.3-10,label={lst:lstlisting}]
"@WWWDecodeDelegate@" -s -k -L -o "%o" "https:%F"
\end{lstlisting}
\vspace{5mm}

Schaut man nun wieder im jeweiligen "`MagickPropertyLetter"' nach, erkennt man, dass der Inhalt von "`F"' nun nicht mehr, wie zuvor, einfach weitergereicht wird, sondern wie in der SanitizeDelegateCommand()-Methode bereinigt, bzw. gefiltert wird.\\
Der Parameter "`M"' wird dabei nicht einfach ersetzt, sondern bleibt in der vorherigen Form vorhanden, da noch andere Delegates darauf zugreifen.\\

\begin{lstlisting}[firstnumber=2610, language=C, caption=magick/property.c Gefilterte Wietergabe F-Parameter,label={lst:lstlisting}]
  case 'F':
  {
    const char
      *q;

    register char
      *p;

    static char
      whitelist[] =
        "^-ABCDEFGHIJKLMNOPQRSTUVWXYZabcdefghijklmnopqrstuvwxyz0123456789"
        "+&@#/%?=~_|!:,.;()";

    /*
      Magick filename (sanitized) - filename given incl. coder & read mods.
    */
    (void) CopyMagickString(value,image->magick_filename,MaxTextExtent);
    p=value;
    q=value+strlen(value);
    for (p+=strspn(p,whitelist); p != q; p+=strspn(p,whitelist))
      *p='_';
    break;
  }
\end{lstlisting}
\vspace{5mm}

Hier fällt auf, dass die Anführungszeichen " und ' nicht mehr in der Whitelist enthalten sind.\\

Dadurch wird nicht mehr vorzeitig vor dem Ende des Bildinhalts aus dem String ausgebrochen und so kein Kommando mehr irrtümlich an die Shell weitergegeben.