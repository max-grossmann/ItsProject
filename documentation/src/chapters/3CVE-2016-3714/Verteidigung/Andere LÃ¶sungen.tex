
\subsubsection{Andere Lösungen}

Nach Bekanntwerden der Schwachstelle wurden schnelle Lösungen veröffentlicht, die nicht offiziell von ImageMagick stammen, jedoch das Ausnutzen der Schwachstelle verhindern.\\

So wird empfohlen, die Magic-Bytes zu überprüfen~\cite{ImageTragick}.\\
Dies muss mit überschaubarem Aufwand im Code getan werden, ist jedoch für den Laien nicht ohne Weiters möglich.\\
Jedoch kann so sicher gestellt werden, dass die Datei zumindest schon einmal das richtige Format hat und ein extrem simpler Angriff vermieden werden.\\

Der empfohlene Fix ist,  die Datei config/Policy.xml anzupassen~\cite{ImageTragick}.\\
Diese umfasst folgende Einträge:\\

\begin{lstlisting}[firstnumber=47, language=XML, caption=config/Policy.xml Inhalt,label={lst:lstlisting}]
<policymap>
<!-- <policy domain="resource" name="temporary-path" value="/tmp"/> -->
<!-- <policy domain="resource" name="memory" value="2GiB"/> -->
<!-- <policy domain="resource" name="map" value="4GiB"/> -->
<!-- <policy domain="resource" name="width" value="10MP"/> -->
<!-- <policy domain="resource" name="height" value="10MP"/> -->
<!-- <policy domain="resource" name="area" value="1GB"/> -->
<!-- <policy domain="resource" name="disk" value="16EB"/> -->
<!-- <policy domain="resource" name="file" value="768"/> -->
<!-- <policy domain="resource" name="thread" value="4"/> -->
<!-- <policy domain="resource" name="throttle" value="0"/> -->
<!-- <policy domain="resource" name="time" value="3600"/> -->
<!-- <policy domain="system" name="precision" value="6"/> -->
<policy domain="cache" name="shared-secret" value="passphrase"/>
</policymap>
\end{lstlisting}
\vspace{5mm}

Hier sind standardmäßig keine weiteren Einschränkungen eingetragen, obwohl die verantwortlichen Coder deaktiviert werden könnten:\\

\begin{lstlisting}[language=XML, caption=config/Policy.xml Inhalt,label={lst:lstlisting}]
<policymap>
  <policy domain="coder" rights="none" pattern="EPHEMERAL" />
  <policy domain="coder" rights="none" pattern="URL" />
  <policy domain="coder" rights="none" pattern="HTTPS" />
  <policy domain="coder" rights="none" pattern="MVG" />
  <policy domain="coder" rights="none" pattern="MSL" />
  <policy domain="coder" rights="none" pattern="TEXT" />
  <policy domain="coder" rights="none" pattern="SHOW" />
  <policy domain="coder" rights="none" pattern="WIN" />
  <policy domain="coder" rights="none" pattern="PLT" />
</policymap>
\end{lstlisting}
\vspace{5mm}

Somit hat der verantwortliche https-Delegate keine Rechte mehr und wird geblockt.\\
Dadurch wird die Sicherheitslücke zwar blockiert, jedoch ist es auch nicht mehr möglich, ImageMagick zur Bearbeitung eines Bildes mit URL zu benutzen.\\