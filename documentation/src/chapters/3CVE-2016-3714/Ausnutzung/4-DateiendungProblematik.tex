\newpage
\subsection{Die Problemematik der Datei Endung}\label{subsec:die-problemematik-der-datei-endung}
Datei-Endungen werden benutzt, damit Menschen direkt wissen, um welchen Dateityp es sich handelt.
Außerdem hat es den Vorteil, dass im Betriebssystem für Datei-Endungen ein gewisses Standard-Program festgelegt werden kann.
So können z.B. .html Dateien standardmäßig mit Firefox oder .txt Dateien standardmäßig mit dem Editor geöffnet werden.

Für Imagemagick ist die Dateiendung irrelevant.
Bild-Typen werden anhand des Dateiinhalts, nicht der Endung im Dateinamen erkannt.~\cite{ImageTragick}
Dies ist problematisch, da hier auch der User getäuscht werden kann.
Durch das Vorschaubild der gewohnten Bildvorschauanwendung, verlässt sich der User darauf, dass eine Datei mit der Endung .png auch wirklich vom Typ PNG ist.
Allerdings kann es sich z.B. auch um eine infizierte MVG-Datei handeln.
Dies ist vor allem im gleich beschriebenen Social Engineering Fall relevant.