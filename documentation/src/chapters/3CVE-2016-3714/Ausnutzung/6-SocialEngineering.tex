\newpage
\subsection{Social Engineering}\label{subsec:social-engineering}

Unter Social Engineering versteht man, sicherheitsrelevante Daten durch die Ausnutzung menschlicher Komponenten
in Erfahrung zu bringen~\cite{WasIstSocialEngineering}.\\\\

Ein folgendes Szenario wäre denkbar:\\

Die Marketingfirma M kümmert sich um die Website der Firma X. Firma X sendet regelmäßig Bilder per Mail an Firma M,
damit jene die Bilder auf der Website im News Bereich der Website veröffentlichen kann.\\

Bilder werden über ein CMS verwaltet.
Es werden also zur Pflege der Websites keine Informatiker mit Erfahrung in IT-Sicherheit benötigt.\\

Die Bilder sind teilweise in sehr hoher Auflösung fotografiert worden, sind also teilweise einzeln über 20MB groß.
Damit die Bilder schneller hochgeladen und den Besuchern der Website eine gute User Experience
durch schnelle Ladezeit geboten werden kann, müssen diese Bilder vor dem Upload noch verkleinert werden.\\

Die IT-Abteilung der Firma M, hat Imagemagick installiert und den Mitarbeitern ein Program geschrieben,
bei welchem nur der Dateiname übergeben werden muss.
Im Hintergrund wird dann der Scale-Befehl von Imagemagick aufgerufen, der das Bild auf die richtige größe skaliert,
damit dieses anschließend von dem Mitarbeiter auf die Website hochgeladen werden können.\\

Ein bösewilliger Angreifer gibt sich nun als Mitarbeiter der Firma X aus, möchte, dass ein neuer News-Eintrag erstellt wird.
Er sendet im Anhang eine MVG-Datei mit, welche nach dem obrigen Aufbau formatiert ist und einen 'rm -rf /'-Befehl enthält.
Die MVG-Datei hat die Endung .png, wodurch die Datei für den Mitarbeiter ungefährlich aussieht.\\

Der Mitarbeiter, welcher die Mail bearbeitet, erkennt diese nicht als Schadsoftware
und führt das Program zum Skalieren von Dateien mit dieser Datei als Parameter aus.
Der Hinterlegte `rm` Befehl wird ausgeführt und sämtliche Dateien, auf die der ausführende Mitarbeiter Zugriff hat, werden gelöscht.
Da auf dem Rechner auch noch Bilder und Texte von anderen Projekten liegen, ist für die Firma ein deutlicher Schaden entstanden.
Bilder müssen aus Backups wieder hergestellt werden
und gewisse Dateien müssen neu angefordert beziehungsweise neu erarbeitet werden,
was einige Arbeitstage für die komplette Firma in Anspruch nimmt.\\

Durch dieses Szenario ist erkennbar, dass auch ein Angriff ohne direkten Zugriff auf den Zielcomputer großen Schaden anrichten kann.