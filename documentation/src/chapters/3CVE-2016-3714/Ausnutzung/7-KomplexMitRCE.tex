
\subsection{Komplexes Beispiel mit Remote Code Execution}\label{subsec:komplexes-beispiel-mit-remote-code-execution}

\subsubsection{Erklärung}
In dem nachfolgenden Beispiel soll eine Situation gezeigt werden,
in der der Angreifer von einem Server mit öffentlicher IP aus direkt die Möglichkeit hat,
sensible Daten abzugreifen und potentiell auch Schaden auf dem Zielserver anzurichten kann.\\

Für das Beispiel soll ein Forum simuliert werden, bei dem User ein eigenes Profilbild hochladen können.
Da Profilbilder im Forum nur sehr klein angezeigt werden müssen,
sollen diese per Imagemagick herunterskaliert werden,
um Speicherplatz zu sparen und damit die Bilder schneller an den Website-Aufrufer ausgeliefert werden kann.
Besonders für User, die über Mobile Daten zugreifen,
ist jede Optimierung der zu übertragenen Bildern und Source-Dateien sehr wichtig.\\

Bei dieser Situation handelt es sich um einen typischen Use-Case,
da in einem Forum Nutzern immer die Möglichkeit haben, selbst Bilder hochzuladen.
Diese Bilder müssen, wie beschrieben analysiert und modifiziert werden,
was oft die Software ImageMagick übernimmt.