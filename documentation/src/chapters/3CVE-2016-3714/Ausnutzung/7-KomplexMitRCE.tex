
\subsection{Komplexes Beispiel mit Remote Code Execution}\label{subsec:komplexes-beispiel-mit-remote-code-execution}

\subsubsection{Erklärung}
In dem nachfolgenden Beispiel soll eine Situation gezeigt werden, in der der Angreifer von seinem PC aus direkt die Möglichkeit hat Schaden auf dem Zielserver anzurichten und sensible Daten abzugreifen.\\

Für das Beispiel soll ein Forum simuliert werden, bei dem User ein eigenes Profilbild hochladen können.
Da Profilbilder im Forum nur sehr klein angezeigt werden müssen, sollen diese per Imagemagick herunterskalliert werden, um Speicherplatz zu sparen und damit die Bilder schneller an den Website-Aufrufer ausgeliefert werden kann.
Besonders für User, die über Mobile Daten zugreifen, ist jede Optimierung der zu übertragenen Bildern und Source-Dateien sehr wichtig.