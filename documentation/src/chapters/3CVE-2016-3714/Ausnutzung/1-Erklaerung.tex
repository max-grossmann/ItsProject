\subsection{Erklärung und einfache Beispiele}\label{subsec:erklaerung-und-einfache-beispiele}

Bei den folgenden Beispielen wird eine Shell-Verbindung zum Zielsystem benötigt.
Es wird jeweils eine MVG Datei erstellt und diese anschließend mit dem Imagemagick-Befehl ′identify′ ausgeführt~\cite{PHPImagickIdentifyImage}.
Dieser Befehl wird normalerweise dafür benutzt,
um Informationen über ein Bild - wie die Bildgröße oder den Bildtyp - zu bekommen.
Die Sicherheitslücke ist jedoch nicht nur auf diesen Befehl begrenzt.\\

\begin{lstlisting}[language=Bash, caption=Erklaerung - Identify einer validen PNG Datei,label={lst:lstlisting}]
> identify valid.png
valid.png PNG 320x240 320x240+0+0 8-bit sRGB 2c 302B 0.000u 0:00.000
\end{lstlisting}

