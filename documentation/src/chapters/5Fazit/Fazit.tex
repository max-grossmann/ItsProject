\chapter{Fazit}\label{ch:fazit}

Die Schwachstelle ist prinzipiell sehr simpel.\\
Da der Input nicht sinnvoll überprüft wird, also nicht sicher gestellt wird, dass zum Beispiel eine .png-Datei auch wirklich nur eine .png-Date ist, ist ein tiefgreifender Angriff auf das System möglich.\\

Die Behebung der Schwachstelle ist ähnlich simpel, wie die Schwachstelle selbst - der Input wird nun gefiltert und die Anführungszeichen, durch die das Ausbrechen aus dem String möglich ist, entfernt.\\

Alternativ wird empfohlen, die Magic-Bytes zu überprüfen.\\
So kann sicher gestellt werden, dass die Datei zumindest schon einmal das richtige Format hat und ein exterm simpler Angriff vermieden werden.\\

Es ist aber auch möglich, die Datei config/Policy.xml anzupassen.\\
Diese umfasst folgende Einträge:\\

\begin{lstlisting}[firstnumber=47, language=XML, caption=config/Policy.xml Inhalt,label={lst:lstlisting}]
<policymap>
<!-- <policy domain="resource" name="temporary-path" value="/tmp"/> -->
<!-- <policy domain="resource" name="memory" value="2GiB"/> -->
<!-- <policy domain="resource" name="map" value="4GiB"/> -->
<!-- <policy domain="resource" name="width" value="10MP"/> -->
<!-- <policy domain="resource" name="height" value="10MP"/> -->
<!-- <policy domain="resource" name="area" value="1GB"/> -->
<!-- <policy domain="resource" name="disk" value="16EB"/> -->
<!-- <policy domain="resource" name="file" value="768"/> -->
<!-- <policy domain="resource" name="thread" value="4"/> -->
<!-- <policy domain="resource" name="throttle" value="0"/> -->
<!-- <policy domain="resource" name="time" value="3600"/> -->
<!-- <policy domain="system" name="precision" value="6"/> -->
<policy domain="cache" name="shared-secret" value="passphrase"/>
</policymap>
\end{lstlisting}
\vspace{5mm}

Hier sind standardmäßig keine weiteren Einschränkungen eingetragen, obwohl die verantwortlichen Coder deaktiviert werden könnten:\\
//Hier Quellenverweis imagetragick!!!

\begin{lstlisting}[language=XML, caption=config/Policy.xml Inhalt,label={lst:lstlisting}]
<policymap>
  <policy domain="coder" rights="none" pattern="EPHEMERAL" />
  <policy domain="coder" rights="none" pattern="URL" />
  <policy domain="coder" rights="none" pattern="HTTPS" />
  <policy domain="coder" rights="none" pattern="MVG" />
  <policy domain="coder" rights="none" pattern="MSL" />
  <policy domain="coder" rights="none" pattern="TEXT" />
  <policy domain="coder" rights="none" pattern="SHOW" />
  <policy domain="coder" rights="none" pattern="WIN" />
  <policy domain="coder" rights="none" pattern="PLT" />
</policymap>
\end{lstlisting}
\vspace{5mm}

Letzten Endes ist die Schwachstelle nur ein weiterer Hinweis und kann als Aufruf dafür verstanden werden, Computersysteme immer aktuell zu halten.\\
Werden die Betriebssysteme regelmäßig geupdatet, vor allem wenn der Support ausläuft, lassen sich solche Schwachstellen vermeiden.\\
Dabei hilft es auch, regelmäßig in Changelogs zu schauen um zu erfahren, ob manuelle Anpassungen in der Software nötig sind, wie zum Beispiel in der Policy.xml.\\

Andernfalls kann ein hilfreiches und einfach zu handhabendes Programm wie ImageMagick sehr schnell zu einem großen Problem werden.