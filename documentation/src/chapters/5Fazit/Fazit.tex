\chapter{Fazit}\label{ch:fazit}

Die Schwachstelle ist auf Grund eines einfachen Fehlers entstanden.\\
Da der Input nicht sinnvoll überprüft wird, also nicht sicher gestellt wird, dass zum Beispiel eine .png-Datei auch wirklich nur eine .png-Datei ist, ist ein tiefgreifender Angriff auf das System möglich.\\

Die Behebung der Schwachstelle ist möglich, indem der Input gefiltert wird und die Anführungszeichen, durch die das Ausbrechen aus dem String möglich ist, entfernt werden.\\

Letzten Endes ist die Schwachstelle nur ein weiterer Hinweis und kann als Aufruf dafür verstanden werden, Computersysteme immer aktuell zu halten.\\
Werden die Betriebssysteme regelmäßig geupdatet, vor allem wenn der Support ausläuft, lassen sich solche Schwachstellen vermeiden.\\
Dabei hilft es auch, regelmäßig in Changelogs zu schauen um zu erfahren, ob manuelle Anpassungen in der Software nötig sind, wie zum Beispiel in der Policy.xml.\\

Andernfalls kann ein hilfreiches und einfach zu handhabendes Programm wie ImageMagick sehr schnell zu einem großen Problem werden.