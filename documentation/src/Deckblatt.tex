

	\begin{center}

		\tikz [remember picture, overlay] %
		\node [shift={(2cm,-2cm)}] at (current page.north west) %
		[anchor=north west] %
		{\includegraphics[scale=0.75]{img/Logo.jpg}};

		\vfill

			\LARGE{Studienarbeit über die Schwachstelle \\CVE-2016-3714}

		\vfill

			\large{Max Großmann (301118) und André Klein (359618)\\
			Hochschule für angewandte Wissenschaften Hof}

		\vfill

			\large{IT-Sicherheit\\
			Wintersemester 2020/2021}

			\vfill
\vfill
			\normalsize{\textbf{Zusammenfassung:}
\vfill
			\begin{justify}
				Die Studienarbeit im Fach IT-Sicherheit befasst sich mit der Schwachstelle CVE-2016-3714 in der Software ImageMagick aus dem Jahr 2016.
				Diese Schwachstelle erlaubt es, eine Remote-Code-Execution auf dem angegriffenen System auszuführen, bei der durch fehlende Überprüfung des Inputs der Angriffscode in einem bearbeiteten Bild direkt im Terminal ausgeführt werden kann.
				Aufgrund des großen Marktanteils und der Beliebtheit von ImageMagick als schnelles und einfaches Bildbearbeitungstool ist diese Schwachstelle kritisch und wurde als "`ImageTragick"' bekannt.}
			\end{justify}
\vfill
\end{center}

